\documentclass[11pt]{article}

\usepackage{fullpage}
\usepackage{amsthm,amssymb,amsmath}  
\usepackage{xspace,enumerate}
\usepackage[utf8]{inputenc}
\usepackage{thmtools}
\usepackage{thm-restate}
\usepackage{authblk}
\usepackage[hypertexnames=false,colorlinks=true,urlcolor=Blue,citecolor=Green,linkcolor=BrickRed]{hyperref}

\newcommand{\myparagraph}[1]{\paragraph{#1.}}

 % \theoremstyle{plain}
  \newtheorem{theorem}{Theorem}
  \newtheorem{lemma}[theorem]{Lemma}  
  \newtheorem{corollary}[theorem]{Corollary}  
  \newtheorem{proposition}[theorem]{Proposition}  
  \newtheorem{fact}[theorem]{Fact}
  \newtheorem{observation}[theorem]{Observation}
  \newtheorem{definition}[theorem]{Definition}
  \newtheorem{example}[theorem]{Example}
  \newtheorem{remark}[theorem]{Remark}
 
  \newtheorem*{claim}{Claim}
 

\author[1,2]{Panagiotis Charalampopoulos}
\author[3]{Paweł Gawrychowski}
\author[2]{Shay Mozes}
\author[4]{Oren Weimann}


\affil[1]{
Department of Informatics, King's College London, UK\\
\href{mailto:panagiotis.charalampopoulos@kcl.ac.uk}{panagiotis.charalampopoulos@kcl.ac.uk}}

\affil[2]{
Efi Arazi School of Computer Science, The Interdisciplinary Center Herzliya, Israel\\
\href{mailto:smozes@idc.ac.il}{smozes@idc.ac.il}}

\affil[3]{
Institute of Computer Science, University of Wrocław, Poland\\
\href{mailto:gawry@mimuw.edu.pl}{gawry@mimuw.edu.pl}}

\affil[4]{
Department of Computer Science, University of Haifa, Israel\\
\href{mailto:oren@cs.haifa.ac.il}{oren@cs.haifa.ac.il}
}


\date{\vspace{-5ex}}


\usepackage[dvipsnames,usenames]{color}


\usepackage[ruled]{algorithm}
\usepackage[noend]{algpseudocode}
\algnewcommand{\LineComment}[1]{\State \(\triangleright\) #1}

\usepackage[margin=1in]{geometry}
\usepackage{graphicx}
\usepackage[noadjust]{cite}
  \usepackage{microtype}
    \usepackage{xspace}
    \usepackage{amsmath,amsfonts}
    \usepackage{wasysym}
  \usepackage{makecell}
  \usepackage{tabularx}
  \usepackage{comment}
  \usepackage{todonotes}
  \usepackage{thmtools}
  \usepackage{thm-restate}
  \usepackage[capitalise]{cleveref}
  \usepackage{verbatim}
\usepackage{units}
\usepackage{color}
\definecolor{darkblue}{rgb}{0,0.08,0.45}
  \usepackage{epsfig}
    \usepackage{graphics}
\usepackage{pgf,tikz}
  \usetikzlibrary{trees, arrows, shapes, snakes}
\usepackage{marvosym}
  \usepackage[caption=false]{subfig}
\usetikzlibrary{decorations.pathmorphing}
\usetikzlibrary{decorations.markings}
\usetikzlibrary{decorations.pathmorphing,shapes}
\usetikzlibrary{calc,decorations.pathmorphing,shapes}
\usepackage{subfig}
\usepackage{booktabs}
\usepackage{etoolbox}
\usepackage[title]{appendix}
\usepackage{enumitem}


\newif\iffull
\fulltrue

\newif\ifspreport
\spreporttrue

\DeclareMathOperator*{\argmax}{argmax}
\DeclareMathOperator*{\argmin}{argmin}

\newcommand{\cO}{\mathcal{O}}
\newcommand{\TG}{\mathcal{T}}
\newcommand{\cOtilde}{\tilde{\cO}}
\newcommand{\Vor}{\textsf{Vor}}
\newcommand{\VD}{\textsf{VD}}
\newcommand{\out}[1]{#1^{out}}
\newcommand{\VDin}{\textsf{VD}_{in}}
\newcommand{\VDout}{\textsf{VD}_{out}}
\newcommand{\weight}{{\rm \omega}}


\begin{document}

\title{Fault-Tolerant Distance Labeling for Planar Graphs\thanks{This work was partially supported by the Israel Science Foundation under grant number 592/17.}}
\maketitle

%\thispagestyle{empty}

\begin{abstract}
In fault-tolerant distance labeling we wish to assign short labels to the vertices of a graph $G$ such that from the labels of any three vertices $u,v,w$ we can infer the $u$-to-$v$ distance in the graph $G\setminus \{v\}$. We show that any directed weighted planar graph admits labels of size\footnote{The $\cOtilde(\cdot)$ notation hides polylogarithmic factors.} $\cOtilde(n^{2/3})$. \todo{multiple faults, approximate distance labeling (improve query time in Chechick et al}
\end{abstract}


%\clearpage
%\setcounter{page}{1}

\section{Introduction}

\paragraph{Exact distance labeling.}
Shay's SODA'19 paper and citations within

\paragraph{Approximate distance labeling.}
In~\cite{DBLP:conf/stoc/AbrahamCG12} it was shown that for
edge weights in $[1, M]$ and parameter $\epsilon  > 0$ there is a labeling scheme with labels of size 
$\lambda = O(\epsilon^{-1}\log^2 n\log(nM) \cdot (\epsilon^{-1}
+ \log n))$ such that from the labels of any two vertices $s$ and $t$ and of any set $F$ of faulty vertices/edges we can deduce the distance between $s$ and $t$ in $G \setminus F$ with stretch $(1 + \epsilon)$ in $O(|F|^2\cdot\lambda)$ time. In \cref{sec:approx} we show how to improve the query time to $O(|F| \cdot\lambda)$.





%\paragraph{Distance oracles for planar graphs.}
%Distance oracles for planar graphs have been extensively studied. Works on exact distance oracles for planar graphs started in the 1990's with oracles requiring $\cOtilde(n^2/Q)$ space and $\cOtilde(Q)$ query-time for any $Q \in [1,n]$~\cite{DBLP:conf/wg/Djidjev96,DBLP:conf/esa/ArikatiCCDSZ96}. 
%Note that this includes the two trivial approaches mentioned above.  
%Over the past three decades, many other works presented exact distance oracles for planar graphs with increasingly better space to query-time tradeoffs~\cite{DBLP:conf/wg/Djidjev96,DBLP:conf/esa/ArikatiCCDSZ96,DBLP:conf/stoc/ChenX00,FR,DBLP:conf/wads/Nussbaum11,DBLP:conf/soda/MozesS12,DBLP:journals/algorithmica/Cabello12,DBLP:conf/focs/Cohen-AddadDW17,vorexact}. 
%
%Until recently, no distance oracles with subquadratic space and polylogarithmic query time were known. 
%Cohen-Addad et al.~\cite{DBLP:conf/focs/Cohen-AddadDW17}, inspired by Cabello's use of Voronoi diagrams for the diameter problem in planar graphs~\cite{DBLP:conf/soda/Cabello17}, provided the first such oracle. 
%The currently best known tradeoff~\cite{vorexact} is an oracle with  $\cOtilde(n^{3/2}/Q)$ space and $\cOtilde(Q)$ query-time for any $Q \in [1,n^{1/2}]$.
%Note that all known oracles with nearly linear (i.e. $\cOtilde(n)$) space require $\Omega(\sqrt{n})$ time to answer queries.\todo{add our new oracle}
%
% 
%\paragraph{Our results and techniques.}
%

\section{Preliminaries} \label{sec:prelim}

Throughout the paper we consider as input a weighted directed planar graph $G=(V(G),E(G))$, embedded in the plane.
The dual of a planar graph $G$ is another planar graph $G^*$ whose vertices correspond to faces of $G$ and
vice versa. Arcs of $G^*$ are in one-to-one correspondence with arcs of $G$; 
there is an arc $e^*$ from vertex $f^*$ to vertex $g^*$ of $G^*$
if and only if the corresponding faces $f$ and $g$ of $G$ are to the left and right of the arc $e$, respectively.


We assume that the input graph has no negative length cycles.
We can transform the graph in a standard~\cite{DBLP:conf/esa/MozesW10} way so that all edge weights are non-negative and distances are preserved.
With another standard transformation we can guarantee, in $\cO(n \frac{\log^2 n}{\log \log n})$ time, that each vertex has constant degree and that the graph is triangulated, while distances are preserved and without increasing asymptotically the size of the graph.
We further assume that shortest paths are unique; this can be ensured in $\cO(n)$ time by a deterministic perturbation of the edge weights~\cite{DBLP:conf/stoc/0001FL18}.
Let $d_G(u,v)$ denote the distance from a vertex $u$ to a vertex $v$ in $G$.

\paragraph{Separators and recursive decompositions.}
Miller~\cite{DBLP:conf/stoc/Miller84} showed how to compute, in a biconnected triangulated planar graph with $n$ vertices, a simple cycle of size  $\cO(\sqrt{n})$ that separates the graph into two subgraphs, each with at most $2n/3$ vertices. Simple cycle separators  can be used to recursively separate a planar graph until pieces have constant size.
The authors of~\cite{DBLP:conf/stoc/KleinMS13} show how to obtain a complete recursive decomposition tree $\TG$ of $G$ in $\cO(n)$ time. 
$\TG$ is a binary tree whose nodes correspond to subgraphs of $G$ (pieces), with the root being all of $G$ and the leaves being pieces of constant size.
We identify each piece $P$ with the node representing it in $\TG$. We can thus abuse notation and write $P\in \TG$.
An $r$-division~\cite{DBLP:journals/siamcomp/Frederickson87} of a planar graph, for $r \in [1,n]$, is a decomposition of the graph into $\cO(n/r)$ pieces, each of size $\cO(r)$, such that each piece has $\cO(\sqrt{r})$ boundary vertices, i.e.~vertices that belong to some separator along the recursive decomposition used to obtain $P$.
Another desired property of an $r$-division is that the boundary vertices lie on a constant number of faces of the piece (holes).
For every $r$ larger than some constant, an $r$-division with few holes is represented in the decomposition tree $\TG$ of~\cite{DBLP:conf/stoc/KleinMS13}. In fact, it is not hard to see that if the original graph $G$ is triangulated then all vertices of each hole of a piece are boundary vertices.
Throughout the paper, to avoid confusion, we use ``nodes" when referring to $\TG$ and ``vertices" when referring to $G$.
We denote the boundary vertices of a piece $P$ by $\partial P$. 
%We refer to non-boundary vertices as internal.
%We assume for simplicity that each hole is a simple cycle. Non-simple cycles do not pose a significant obstacle, as we discuss  at the end of~\cref{sec:onehole}.
%
%It is shown in~\cite[Theorem 3]{DBLP:conf/stoc/KleinMS13} that, given a geometrically decreasing sequence of numbers $(r_m, r_{m-1}, \ldots, r_1)$, where $r_1$ is a sufficiently large constant, $r_{i+1}/r_i=b$ for all $i$ for some $b>1$, and $r_m=n$, we can obtain the $r_i$-divisions for all $i$ in time $\cO(n)$ in total.
%For convenience we define the only piece in the $r_m$ division to be $G$ itself.
%These $r$-divisions satisfy the property that a piece in the $r_i$-division is a ---not necessarily strict--- descendant (in $\TG$) of a piece in the $r_j$-division for each $j>i$. This ancestry relation between the pieces of an $r$-division can be captures by a tree $\TG$ called the recursive $r$-division tree.

The boundary vertices of a piece $P \in \TG$ that lie on a hole $h$ of $P$ separate the graph $G$ into two subgraphs $G_1$ and $G_2$ (the cycle is in both subgraphs). One of these two subgraphs is enclosed by the cycle and the other is not. Moreover, $P$ is a subgraph of one of these two subgraphs, say $G_1$. We then call $G_2$ the outside of hole $h$ with respect to $P$ and denote it by $P^{h,out}$.
In the sections where we assume that the boundary vertices of each piece lie on a single hole that is a simple cycle, the outside of this hole with respect to $P$ is $G\setminus(P \setminus \partial P)$ and to simplify notation we denote it by $\out{P}$. 


\section{Exact distance labeling}\label{sec:exact}
Each vertex $v$ stores the distances among boundary nodes of its piece avoiding it. Also, each vertex stores for each piece in the complete recursive decomposition above the $r$-division, the distances from/to the boundary of the piece avoiding the sibling of the piece (to be clear - also avoiding the boundary of the sibling).  Overall $r + n/\sqrt{r}$ balancing at $r = n^2/3$.
\todo{multiple faults}

\section{Approximate distance labeling}\label{sec:approx}
In this section we show how to improve the query time in~\cite{DBLP:conf/stoc/AbrahamCG12}

\bibliographystyle{plain}
\bibliography{references}
\end{document}